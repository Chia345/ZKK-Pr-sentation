\begin{frame}[fragile]{Zuweisung}
\begin{itemize}
	\item Zuweisung von Variablen mit dem Zuweisungsoperator \texttt{=}
	\begin{lstlisting}
	a = 5
	b = 3.14
	c = "Hallo Grundkurs:Programmieren"
	\end{lstlisting}
	\item der Variable kann auch das Ergebnis einer Operation zugewiesen werden
	\begin{lstlisting}
	a = 1000
	b = 200
	percent = b / a * 100
	\end{lstlisting}
	\item mehrfache Zuweisung 
	\begin{lstlisting}
	a, b, c = 5, 3.14, "Hallo Grundkurs:Programmieren"
	x = y = z = 42
	\end{lstlisting}
\end{itemize}
\end{frame}

\begin{frame}[fragile]{Typen und Variablen}
   
    \begin{block}{Aufgabe}
       Lasse dich von deinem Programm begrüßen, indem du mit \texttt{input} deinen Namen als  
       Eingabe in einer Variable speicherst.
    \end{block}
    \pause{}
    \begin{exampleblock}{Lösung}
        \begin{lstlisting}
>>> name = input()
'Maren'
>>> print("Hallo " + name)
        \end{lstlisting}
    \end{exampleblock}
\end{frame}

\begin{frame}[fragile]{Typconversion}
    \begin{alertblock}{Achtung}
    Die \lstinline{input()} Funktion interpretiert jede Benutzereingabe
    als \texttt{String}. Wenn man Zahlen aufnehmen will, muss der Typ
    `gecastet' werden, d.h. `umgewandelt'.
    \begin{itemize}
        \item \lstinline{int()}: Castet zu int.
        \item \lstinline{str()}: Castet zu String.
    \end{itemize}
    Was passiert bei \lstinline{int("HalloWelt")}?
    \pause{}
    Ein \texttt{ValueError} wird geworfen. (Exception handling nicht 
    in diesem Kurs)
    \end{alertblock}
\end{frame}
