\begin{frame}[fragile]{Operatoren}
	\begin{itemize}
		\item Rechenoperatoren 
		\item vergleichende Operatoren
	\end{itemize}
\end{frame}

%Einführung Rechenoperationen
\begin{frame}[fragile]{Rechenoperatoren}
\begin{itemize}
	\item + und -
	\item * und /
	\item Modulo \%
\end{itemize}
\end{frame}


%Beispielaufgabe mit print
\begin{frame}[fragile]{Zahlen und Rechenoperatoren}
\begin{lstlisting}
print(2.45 + 3)
print("Hallo" + 3)
print(1/2.5 +2)
print(3%2)
\end{lstlisting}
\end{frame}


\begin{frame}[fragile]{Typsicherheit}
\begin{block}{Aufgabe}
	Gib folgende Ausdrücke in den Python Interpreter ein:
	\begin{lstlisting}
	>>> 3 + 3.14
	>>> "Mein Alter: " + 5
	>>> True + 1
	\end{lstlisting}
\end{block}
\pause{}
\begin{exampleblock}{True + 1 == 2?}
	Intern werden die Keywords \lstinline{True} und \lstinline{False} auf die 
	Werte \texttt{1} und \texttt{0} vom Typ \texttt{int} abgebildet.
\end{exampleblock}
\end{frame}


\begin{frame}[fragile]{vergleichende Operatoren}
Die Operatoren \texttt{+}, \texttt{-}, \texttt{/} oder \texttt{*} verwenden wir intuitiv.\\ Wollen wir aber Datentypen vergleichen, benötigen wir weitere Operatoren.\\

\begin{itemize}
	\item \texttt{==}: prüft zwei Werte auf Gleichheit
	\item \texttt{!=}: prüft zwei Werte auf Ungleichheit
	\item \texttt{>}: größer
	\item \texttt{<}: kleiner (vgl.\ for-Schleife)
	\item \texttt{<=, >=}, kleiner-gleich, größer-gleich
	\item \texttt{and}: logisches `Und'
	\item \texttt{or}: logisches `Oder'
	\item \texttt{not}: verneint einen Ausdruck
\end{itemize}
\end{frame}

\begin{frame}[fragile]{vergleichende Operatoren}
Die Operatoren \texttt{+}, \texttt{-}, \texttt{/} oder \texttt{*} verwenden wir intuitiv.\\ Wollen wir aber Datentypen vergleichen, benötigen wir weitere Operatoren.\\

\begin{itemize}
    \item \texttt{==}: prüft zwei Werte auf Gleichheit
    \item \texttt{!=}: prüft zwei Werte auf Ungleichheit
    \item \texttt{>}: größer
    \item \texttt{<}: kleiner (vgl.\ for-Schleife)
    \item \texttt{<=, >=}, kleiner-gleich, größer-gleich
    \item \texttt{and}: logisches `Und'
    \item \texttt{or}: logisches `Oder'
    \item \texttt{not}: verneint einen Ausdruck
\end{itemize}
\end{frame}

\begin{frame}[fragile]{= und ==}
	Der Unterschied zwischen = und == ist sehr wichtig.
	\begin{itemize}
		\item == vergleich beide Seiten und gibt False/True zurück
		\item = ist eine Zuweisung (Lernen wir im nächsten Kapitel kennen)
	\end{itemize}
\end{frame}


\begin{frame}[fragile]{logische Operatoren}
Was ergeben folgende Ausdrücke? Überprüfe mit dem Python Interpreter.

\begin{lstlisting} 
3 > 4 

6 != 7

"Hallo" < "Hallo Welt!"

"Hallo" == "Hallo Welt"

"Hallo Welt" == "Hallo" and 3 > 4
