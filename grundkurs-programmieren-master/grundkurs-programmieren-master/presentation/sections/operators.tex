\section{Operatoren}

\begin{frame}[fragile]{Operatoren}
\begin{itemize}
	\item Rechenoperatoren 
	\item vergleichende Operatoren
	\item logische Operatoren
\end{itemize}
\end{frame}

%Einführung Rechenoperationen
\begin{frame}[fragile]{Rechenoperatoren}
\begin{itemize}
\item + und -
\item * und /
\item Modulo \% (entspricht dem Rest, der durch eine Teilung entsteht)
\end{itemize}
\end{frame}


%Beispielaufgabe mit print
\begin{frame}[fragile]{Zahlen und Rechenoperatoren}
\begin{lstlisting}
print("Ich" + " bin " + str(10) + "Jahre alt")
print("Hallo"*2)
print(2.45 + 3)
print("Hallo" + "3")
print(1/2.5 +2)
print(3%2)
print(6%3)
\end{lstlisting}
\end{frame}



\begin{frame}[fragile]{vergleichende Operatoren}
Wollen wir aber Datentypen vergleichen, benötigen wir weitere Operatoren.\\
Diese ergeben immer einen Booleanwert(True/False)\\

\begin{itemize}
\item \texttt{==} prüft zwei Werte auf Gleichheit
\item \texttt{!=} prüft zwei Werte auf Ungleichheit
\item \texttt{>} größer (bei Strings wird automatisch die Länge vergleichen)
\item \texttt{<} kleiner (bei Strings wird automatisch die Länge vergleichen)
\item \texttt{<=, >=}  kleiner-gleich, größer-gleich (bei Strings wird automatisch die Länge vergleichen)

\end{itemize}
\end{frame}

\begin{frame}[fragile]{vergleichende Operatoren}
Was ergeben folgende Ausdrücke? Überprüfe mit dem Python Interpreter.

\begin{lstlisting} 
3 > 4 

6 != 7

"Hallo" < "Hallo Welt!" 

"Hallo" == "Hallo Welt"

\end{lstlisting}
\end{frame}



\begin{frame}[fragile]{logische Operatoren}
Vergleichen von zwei Wahrheitswerten(meist auf Grundlage von vergleichenden Operatoren)\\
Ergibt immer einen Booleanwert(True/False)

\begin{itemize}	
\item \texttt{and}  logisches `Und' (True, wenn beide Seiten wahr sind)
\item \texttt{or}  logisches `Oder' (True, wenn eine Seite, die andere oder beide wahr sind)
\item \texttt{not}  verneint einen Ausdruck(Verneinung: aus True wird False, aus False wird True)
\end{itemize}
\end{frame}

\begin{frame}[fragile]{logische Operatoren}
Was ergeben folgende Ausdrücke? Überprüfe mit dem Python Interpreter.

\begin{lstlisting} 
3 > 4 or 6 != 7

"Hallo" < "Hallo Welt!" and 3 > 4

not( "Hallo" == "Hallo Welt" )
\end{lstlisting}
\end{frame}







