\begin{frame}[fragile]{Listen}
Listen sind praktische Datenstrukturen, um eine Folge von Werten zu 
speichern oder zu erzeugen.
Oft reichen Integer, Float und String Datentypen nicht aus. Meist wissen wir nämlich im Voraus nicht, 
wie viele Datensätze gespeichert werden sollen.
\pause{}
\begin{lstlisting}
zahlen = [1, 2, 3, 4, 5, 6]
texte = ["Hallo", "Welt", ".", ["Grundkurs", "Programmieren"]]
\end{lstlisting}

\begin{lstlisting}
>>> zahlen[3]
4
>>> texte[0]
"Hallo"
>>> texte[3]
["Grundkurs", "Programmieren"]
\end{lstlisting}
\end{frame}

\begin{frame}[fragile]{Datenstrukturen: Listen}
Die Liste bietet eine große Anzahl an Methoden (Funktionen), die auf ihnen
ausgeführt werden können.
\begin{block}{Aufgabe}
\begin{lstlisting}
>>> liste = ["Grundkurs", "Programmieren", 42, "Pie", 3.14]

>>> liste[2] = 99
>>> len(liste)
>>> liste.append("Passau")
>>> liste.extend([4, 5, 3.14])
>>> liste.insert(2, "Falke")
>>> liste.count(3.14)
>>> liste.index(3.14)
>>> liste.remove(3.14)
>>> liste.pop()
>>> liste.reverse()
>>> sum([1,3,5])
>>> max([1,3,5])
\end{lstlisting}
\end{block}
\end{frame}

\begin{frame}[fragile]{Datenstrukturen: Listen}
\begin{block}{Aufgabe}
Versuche zu erraten, was die Ausgabe dieses Programms ist.
\end{block}

\begin{lstlisting}
liste_a = ["Hallo", "schoenes", "Wetter"]
liste_b = liste_a

liste_b[1] = "schlechtes"

print(liste_a[0], liste_a[1], liste_a[2])  
\end{lstlisting}
\pause{}
\begin{exampleblock}{Lösung}
    \texttt{Hallo schlechtes Wetter}
\end{exampleblock}
\end{frame}

\begin{frame}[fragile]{Datenstrukturen: Listen}
\begin{block}{Aufgabe: Notendurchschnitt}
Schreibe ein Programm, dass drei Prüfungs-Noten einliest, in einer
Liste speichert und dir nach jeder Eingabe den Durschnitt errechnet. 
\end{block}
\begin{lstlisting}

\end{lstlisting}
\pause{}
\begin{exampleblock}{Lösung}
siehe Beamer
\end{exampleblock}
\end{frame}

