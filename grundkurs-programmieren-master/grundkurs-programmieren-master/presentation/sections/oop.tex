\begin{frame}[fragile]{Objektorientierung}
Je größer ein Projekt wird, desto wichtiger ist es, den Überblick 
zu behalten. Funktionen sind eine Art, das Programm übersichtlich zu halten.
Objektorientierung eine weitere.
\pause{}

\begin{lstlisting}
# einfachste Art einer Klasse
class Person:
    pass

james = Person()

james.name = "James"
james.alter = 42
\end{lstlisting}
\end{frame}

\begin{frame}[fragile]{Objektorientierung, die \lstinline{\_\_init\_\_()} methode}

\begin{lstlisting}
class Person():

    def __init__(self):
        self.name = "James"
        self.alter = 42

    def alter_plus_10(self):
        return self.alter + 10

\end{lstlisting}
\end{frame}


\begin{frame}[fragile]{Objektorientierung}
\begin{block}{Aufgabe}
\begin{itemize}
    \item Verpacke das Notenprogramm in eine eigene Klasse Notenprogramm mit dem Attribut noten, in dem
    die Noten gespeichert sind.
    \item Füge die Methode errechne\_durschnitt() hinzu, die den Durchschnitt errechnet
\end{itemize}
\end{block}
\end{frame}


