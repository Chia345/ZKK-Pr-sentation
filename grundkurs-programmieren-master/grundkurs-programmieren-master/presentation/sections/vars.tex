\section{Variablen, Zuweisungen und Typumwandlung}

\begin{frame}[fragile]{Zuweisung, Variablen und Typumwandlung}
\begin{itemize}
	\item Was sind Zuweisungen und Variablen
	\item Anwendung
	\item Nutzen
\end{itemize}
\end{frame}

\begin{frame}[fragile]{Zuweisung}
\begin{itemize}
\item Zuweisung von Variablen mit dem Zuweisungsoperator \texttt{=}
\begin{lstlisting}
a = 5
b = 3.14
c = "Hallo Grundkurs:Programmieren"
\end{lstlisting}
\item der Variable kann auch das Ergebnis einer Operation zugewiesen werden
\begin{lstlisting}
a = 1000
b = 200
percent = b / a * 100
\end{lstlisting}
\item immer sinnvoll benennen

\end{itemize}
\end{frame}

\begin{frame}[fragile]{Zuweisung}

	\begin{lstlisting}
	
	toprint = "Hallo Grundkurs:Programmieren"
	print(toprint)
	Ausgabe:
	
	name = 
	alter =
	print()
	Ausgabe soll sein: `name` ist `alter` Jahre alt
	\end{lstlisting}
	
	\pause{}
	
	\begin{lstlisting}
	toprint = "Hallo Grundkurs:Programmieren"
	print(toprint)
	Ausgabe: "Hallo Grundkurs:Programmieren"
	
	name = Maren
	alter = 23
	print(name + " ist " + str(alter) + "Jahre alt")
	
		\end{lstlisting}
	

\end{frame}


\begin{frame}[fragile]{= und ==}
Der Unterschied zwischen = und == ist sehr wichtig.
\begin{itemize}
\item == Vergleich beider Seiten; gibt False/True zurück
\item = ist eine Zuweisung (Lernen wir im nächsten Kapitel kennen)
\end{itemize}
\end{frame}

\begin{frame}[fragile]{= und ==}
Welche Ausgabe wird folgende Sequenz bringen?
	\begin{lstlisting}
	a = 21
	b = 21
	a == b + 1
	c = a==b
	print(c)
	\end{lstlisting}
\end{frame}


%führt Input() ein
\begin{frame}[fragile]{Eine weitere wichtige Funktion}
input()
\begin{itemize}
\item Liest die letzte Konsolenzeile ein
\item gibt den Konsoleneintrag als String zurück
\item input("") Gibt in der Konsole den Inhalt innerhalb der "" aus bevor eingelesen wird.
\item z.B name = input() 
\end{itemize}
\end{frame}

\begin{frame}[fragile]{Inputaufgabe}   
\begin{block}{Aufgabe}
\begin{itemize}
\item Lasse dich von deinem Programm begrüßen, indem du mit \texttt{input} 
\item "Hallo, wie heißt du" in der die Konsole ausgibst 
\item deinen Namen als  
Eingabe in einer Variable speicherst.
\item "Hallo " und deinen Namen ausgeben lässt
\end{itemize}
\end{block}
\begin{exampleblock}{Lösung}
\begin{lstlisting}
 name = input("Hallo, wie heisst du?")
'Maren'
 print("Hallo " + name)
\end{lstlisting}
\end{exampleblock}
\end{frame}

\begin{frame}[fragile]{Inputaufgabe}   
\begin{block}{Aufgabe}
	\begin{itemize}
		\item Lasse dich von deinem Programm begrüßen, indem du mit \texttt{input} 
		\item "Hallo, wie heißt du" in der die Konsole ausgibst 
		\item deinen Namen als  
		Eingabe in einer Variable speicherst.
		\item Lasse das Programm nach deinem Alter mit ("Wie alt bist du?") fragen
		\item speichere dieses als Integer in einer Variable
		\item erhöhe ihn um 1.
		\item Lasse ausgeben: Du heißt "name" und wirst "alter" Jahre alt
	\end{itemize}
\end{block}
\end{frame}

\begin{frame}[fragile]{Inputaufgabe}
\begin{exampleblock}{Lösung}
	\begin{lstlisting}
	name = input("Hallo, wie heisst du? ")
	alter = int(input("Wie alt bist du? "))
	alter = alter + 1
	print("Du heisst " + name + " und wirst " + str(alter) + " Jahre alt")
	\end{lstlisting}
\end{exampleblock}
\end{frame}
