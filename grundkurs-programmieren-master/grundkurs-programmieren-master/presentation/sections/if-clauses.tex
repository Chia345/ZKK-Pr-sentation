\section{Bedingte Ausführung}

\begin{frame}[fragile]{Wiederholung}
	\begin{itemize}
		\item Vergleich mit Hilfe vergleichender und logischer Operatoren
		\item Vergleiche als Bedingungen
		\item Operatoren
	\begin{itemize}
		\item \texttt{==}: prüft zwei Werte auf Gleichheit
		\item \texttt{!=}: prüft zwei Werte auf Ungleichheit
		\item \texttt{>}: größer
		\item \texttt{<}: kleiner 
		\item \texttt{<=, >=}, kleiner-gleich, größer-gleich
		\item \texttt{and}: logisches `Und'
		\item \texttt{or}: logisches `Oder'
		\item \texttt{not}: verneint einen Ausdruck
	\end{itemize}
	\end{itemize}
\end{frame}
\begin{frame}[fragile]{Übung}
\begin{lstlisting} 
print(4 >= 4) 
print(6 == 7)
print(3 + 4 != 6 or 3 == 5)
not True

\end{lstlisting}
\pause{}
\begin{lstlisting} 
True
False
True or False => True
False
\end{lstlisting}
\end{frame}

\begin{frame}[fragile]{Bedingte Ausführung}
Umgangssprachlich:\\ Wenn (if) eine \textcolor{red}{Bedingung} True ist, dann führe Programmcode 1 aus, andernfalls (else) Programmcode 2\\

\begin{lstlisting}
if Bedingung == True:
	# Programmcode 1 
else:
	# Programmcode 2 
\end{lstlisting}

\begin{lstlisting}
zahl = int(input())

if zahl > 10:
	print("Die Zahl ist > 10.")
else:
	print("Die Zahl ist <= 10.")
\end{lstlisting}
\end{frame}

\begin{frame}[fragile]{Mehrfach bedingte Ausführung}
Umgangssprachlich:\\ Wenn (if) eine \textcolor{red}{Bedingung1} True ist, dann führe Programmcode 1 aus, \\falls nicht dann prüfe (elif) ob \textcolor{red}{Bedingung2} True ist, dann führe Programmcode 2 aus, \\andernfalls (else) Programmcode 3\\

\begin{lstlisting}
if Bedingung == True:
	# Programmcode 1
elif Bedingung2 == True:
	# Programmcode 2
else:
	# Programmcode 3
\end{lstlisting}
\end{frame}

\begin{frame}[fragile]{Mehrfach bedingte Ausführung - Beispiel}
\begin{lstlisting}
zahl = int(input())

if zahl > 10:
	print("Die Zahl ist > 10.")
elif zahl > 5:
	print("Die Zahl ist > 5 und <= 10.")
else:
	print("Die Zahl ist <= 5.")
\end{lstlisting}
\end{frame}

\begin{frame}[fragile]{Schachtelung}
\begin{itemize}
	\item Bedingungen und Schleifen (dazu später) können beliebig oft ineinander geschachtelt werden
	\item Erkennbar durch Einrückungen
	\item Beachte Logik
	\item Zu viele Schachtelungen führen zu Unübersichtlichkeit \texttt{=>} schlechter Code
\end{itemize}
\end{frame}


\begin{frame}[fragile]{Schachtelung bedingter Ausführungen}

\begin{lstlisting}
zahl = int(input())

if zahl < 10:
	if zahl < 5:
		print("Die Zahl ist < 5")
	else:
		print("Die Zahl ist >= 5 und < 10")
else:
	print("Die Zahl ist > 10")
\end{lstlisting}
\end{frame}


\begin{frame}{Bedingte Ausführung - Übung 1}
Aufgabe: Hundealter in Menschenalter\\
Bei kleinen Hunden entspricht das erste Lebensjahr etwa 20 Menschenjahren. Das zweite entspricht 8 Jahren und alle weiteren Hundejahre entsprechen jeweils 4 Menschenjahren. Bei einem 5-jährigen Hund rechnen Sie also: 20 + 8 + 4 + 4 + 4 = 40. Fünf Hundejahre wären demnach etwa 40 Menschenjahre.\\
\end{frame}
\begin{frame}{Bedingte Ausführung - Übung 1}
Kurz:
\begin{itemize}
\item 1 Hundejahr $=$ 20 Jahre
\item 2 Hundejahre $=$ 28 Jahre
\item Über 2 Jahren $= 20 + 8 + (alter - 2)  *  4$ Jahre
\end{itemize}
Aufgabe:
Es soll ein Programm geschrieben werden, dass mit input() nach dem Alter fragt (nur positives Hundealter). Mit bedingter Ausführung das Menschenalter ermittelt und ausgibt.
\begin{itemize}
	\item input("Alter des Hundes: ")
	\item bedingte Ausführung
	\item print("Das entspricht ca. ??? Jahren.")
\end{itemize}

\end{frame}

\begin{frame}[fragile]{Bedingte Ausführung - Übung 1 }
\begin{exampleblock}{Lösung}
\begin{lstlisting}
alter = int(input("Alter des Hundes: "))
if age == 1:
	print("Das entspricht ca. 28 Jahren.")
elif age == 2:
	print("Das entspricht ca. 28 Jahren.")
else:
	human = 28 + (age -2)*4
	print("Das entspricht ca. " + str(human) + " Jahren.")
\end{lstlisting}
\end{exampleblock}
\end{frame}



