\section{Operatoren}

\begin{frame}[fragile]{Operatoren}
\begin{itemize}
	\item Rechenoperatoren 
	\item Vergleichende Operatoren
	\item Logische Operatoren
\end{itemize}
\end{frame}

%Einführung Rechenoperationen
\begin{frame}[fragile]{Rechenoperatoren}
\begin{itemize}
\item \textbf{+} und \textbf{-}
\item \textbf{*} und \textbf{/}
\item Modulo \textbf{\%} (entspricht dem Rest, der durch eine Teilung entsteht)
\end{itemize}
\end{frame}


%Beispielaufgabe mit print
\begin{frame}[fragile]{Zahlen und Rechenoperatoren - Übung}
Was ergeben folgende Ausdrücke? Überprüfe mit  Python.
\begin{lstlisting}
print("Ich" + " bin " + str(10) + " Jahre alt") 
print("Hallo"*2)
print(2.45 + 3)
print("Hallo " + "3")
print(1/2.5 +2)
print(3%2)
print(6%3)
\end{lstlisting}
\pause{}
\begin{lstlisting}
Ich bin 10 Jahre alt
HalloHallo
5.45 
Hallo 3
2.4
1
0
\end{lstlisting}
\end{frame}



\begin{frame}[fragile]{Vergleichende Operatoren}
Wollen wir aber Datentypen vergleichen, benötigen wir weitere Operatoren.\\
Diese ergeben immer einen Booleanwert (True/False).\\

\begin{itemize}
\item \textbf{==} prüft zwei Werte auf Gleichheit
\item \textbf{!=} prüft zwei Werte auf Ungleichheit
\item \textbf{$>$} größer (bei Strings wird automatisch die Länge vergleichen)
\item \textbf{$<$} kleiner (bei Strings wird automatisch die Länge vergleichen)
\item \textbf{$<=$, $>=$}  kleiner-gleich, größer-gleich (bei Strings wird automatisch die Länge vergleichen)

\end{itemize}
\end{frame}

\begin{frame}[fragile]{Vergleichende Operatoren - Übung}
Was ergeben folgende Ausdrücke? Überprüfe mit Python.

\begin{lstlisting} 
print(3 > 4) 

print(6 != 7)

print("Hallo" < "Hallo Welt!") 

print("Hallo" == "Hallo Welt")

\end{lstlisting}
\end{frame}



\begin{frame}[fragile]{Logische Operatoren}
Vergleichen von zwei Wahrheitswerten (meist auf Grundlage von vergleichenden Operatoren)\\
Diese ergeben immer einen Booleanwert (True/False).

\begin{itemize}	
\item \textbf{and}  logisches `Und' (True, wenn beide Seiten wahr sind)
\item \textbf{or}  logisches `Oder' (True, wenn eine Seite, die andere oder beide wahr sind)
\item \textbf{not}  verneint einen Ausdruck (Verneinung: aus True wird False, aus False wird True)
\end{itemize}
\end{frame}

\begin{frame}[fragile]{Logische Operatoren - Übung}
Was ergeben folgende Ausdrücke? Überprüfe mit dem Python Interpreter.

\begin{lstlisting} 
print(3 > 4 or 6 != 7)

print("Hallo" < "Hallo Welt!" and 3 > 4)

print(not( "Hallo" == "Hallo Welt"))
\end{lstlisting}
\end{frame}







