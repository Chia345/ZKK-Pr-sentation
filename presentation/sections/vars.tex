\section{Variablen, Zuweisungen und Typumwandlung}

\begin{frame}[fragile]{Variablen und Zuweisungen}
Lernziele
\begin{itemize}
	\item Kennenlernen von Variablen und Zuweisungen
	\item Variablen und Zuweisungen anwenden

\end{itemize}
\end{frame}

\begin{frame}[fragile]{Variable}
\begin{itemize}
	\item Eine Art Platzhalter/Speicherplatz
	\item Man kann in ihnen Werte speichern
	\item Sie werden kleingeschrieben
	\item Wenn möglich sinnvoll benennen
	\item Bsp. \texttt{name}, \texttt{alter}, \texttt{prozent}, \texttt{age}, \texttt{pi}, \texttt{todelete}
	
\end{itemize}
\end{frame}


\begin{frame}[fragile]{Zuweisung}
\begin{itemize}
\item Zuweisung von Werten zu einer Variablen mit dem Zuweisungsoperator \texttt{=}
\begin{lstlisting}
a = 5
b = 3.14 
c = "Hallo Grundkurs:Programmieren"
\end{lstlisting}
\item Der Variable kann auch das Ergebnis einer Operation zugewiesen werden
\begin{lstlisting}
divisor = 1000
dividend = 200
percent = dividend / divisor * 100
\end{lstlisting}

\end{itemize}
\end{frame}


\begin{frame}[fragile]{Zuweisung - Übung}

Fülle die ... aus
\begin{lstlisting}
toprint = "Hallo"
print(toprint)
Ausgabe: ...
\end{lstlisting}
\begin{lstlisting}
Name = ...
Alter = ...
print(...)
Ausgabe soll sein: Max Mustermann ist 20 Jahre alt
\end{lstlisting}
\pause{}
\begin{lstlisting}
toprint = "Hallo"
print(toprint)
Ausgabe: Hallo

name = "Max Mustermann"
alter = 20
print(name + " ist " + str(alter) + " Jahre alt")
\end{lstlisting}

\end{frame}



\begin{frame}[fragile]{Erneute Zuweisung}
\begin{itemize}
	\item Soll einer Variable ein neuer Wert zugewiesen werden, so ist eine neue Zuweisung mit  \texttt{=} notwendig.
	\item Beispiel: \texttt{a} um 5 erhöhen. Korrigiere den Code
	\begin{lstlisting}
a = 5
a + 5
print(a)
	\end{lstlisting}
	\pause{}
	\begin{lstlisting}
a = 5
a = a + 5
print(a)
	\end{lstlisting}
	
\end{itemize}
\end{frame}






\begin{frame}[fragile]{= und == }
Der Unterschied zwischen = und == ist sehr wichtig.
\begin{itemize}
\item == Vergleich beider Seiten; gibt False/True zurück
\item = ist eine Zuweisung 
\end{itemize}
\end{frame}

\begin{frame}[fragile]{Übung zu = und == }
Welche Ausgabe wird folgendes Programm haben?
	\begin{lstlisting}
a = 21
b = 21
a == b + 1
c = a==b
print(c)
	\end{lstlisting}
\end{frame}


%führt Input() ein
\begin{frame}[fragile]{Eine weitere wichtige Funktion: input()}
input()
\begin{itemize}
\item Liest die letzte Konsolenzeile ein
\item Gibt den Konsoleneintrag als String zurück
\item input(''...'') gibt in der Konsole den Inhalt innerhalb der '''' aus bevor eingelesen wird.
\item z.B name = input("Name: ") 
\end{itemize}
\end{frame}

\begin{frame}[fragile]{Input - Übung}   
\begin{block}{Aufgabe}
\begin{itemize}


\item Lasse dich von deinem Programm begrüßen, indem du
  \texttt{input(''Hallo, wie heißt du? '')} verwendest
\item Deinen Namen als Eingabe in einer Variable speicherst 
\item Lasse ausgeben: \texttt{Hallo }'name'
\end{itemize}
\end{block}
\pause{}
\begin{exampleblock}{Lösung}
\begin{lstlisting}
name = input("Hallo, wie heisst du?")
Eingabe: Maren
print("Hallo " + name)
\end{lstlisting}
\end{exampleblock}
\end{frame}

\begin{frame}[fragile]{Input - Übung}   
\begin{block}{Aufgabe}
	\begin{itemize}
		\item Lasse dich von deinem Programm begrüßen, indem du mit input(''Hallo, wie heißt du? '') ausgeben lässt. (Schon erledigt)
		\item Deinen Namen als Eingabe in einer Variable speicherst. (Schon erledigt)
		\item Lasse das Programm nach deinem Alter mit input(''Wie alt bist du? '') fragen
		\item Speichere die Eingabe als Integer in einer Variable 
		\item Erhöhe das Alter danach um 1
		\item Lasse ausgeben: ''Du heißt 'name' und wirst 'alter' Jahre alt.''
	\end{itemize}
\end{block}
\end{frame}

\begin{frame}[fragile]{Inputaufgabe}
\begin{exampleblock}{Lösung}
	\begin{lstlisting}
name = input("Hallo, wie heisst du? ")
alter = int(input("Wie alt bist du? "))
alter = alter + 1
print("Du heisst " + name + " und wirst " + str(alter) + " Jahre alt.")
	\end{lstlisting}
\end{exampleblock}
\end{frame}
