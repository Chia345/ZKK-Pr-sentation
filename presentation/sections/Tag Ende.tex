\section{Wiederholung}

\begin{frame}[fragile]{Datentypen}
	\begin{itemize}
		\item String (str)
		\item Boolean (bool)
		\item Integer (int)
		\item Float (float)
	\end{itemize}
\end{frame}

\begin{frame}[fragile]{Operatoren}
	\begin{itemize}
		\item \texttt{==}: prüft zwei Werte auf Gleichheit
		\item \texttt{!=}: prüft zwei Werte auf Ungleichheit
		\item \texttt{>}: größer
		\item \texttt{<}: kleiner 
		\item \texttt{<=, >=}, kleiner-gleich, größer-gleich
		\item \texttt{and}: logisches `Und'
		\item \texttt{or}: logisches `Oder'
		\item \texttt{not}: verneint einen Ausdruck
	\end{itemize}
\end{frame}


\begin{frame}[fragile]{Übung}
\begin{lstlisting} 
print(4 >= 4) 
print(6 == 7)
print(3 + 4 != 6 or 3 == 5)
not True
\end{lstlisting}
\pause{}
\begin{lstlisting} 
True
False
True or False => True
False
\end{lstlisting}

\end{frame}

\begin{frame}[fragile]{Variablen und Zuweisungen}
	\begin{itemize}
		\item Werte werden Variablen mit $=$ zugewiesen
		\item sinnvolle, kleingeschriebene Variablennamen
		\item auf Variablentyp achten (welchen Datentyp hat der zugewiesene Wert)
	\end{itemize}
\end{frame}

\begin{frame}[fragile]{Bedingte Ausführung}
	\begin{lstlisting}
	if Bedingung == True:
	     # Programmcode 1
	elif Bedingung2 == True:
	     # Programmcode 2
	...
	else:
	     # Programmcode 3
	\end{lstlisting}
\end{frame}

\begin{frame}[fragile]{Schleifen}
\textbf{for-Schleife}
\begin{lstlisting}
for Variable in range(Anfang, Ende):
		# Programmcode
\end{lstlisting}

\textbf{while-Schleife}
\begin{lstlisting}
while Bedingung == True:
# Programmcode

\end{lstlisting}
\end{frame}