\begin{frame}[fragile]{Allgemeines zu Python}
Kommentare
\begin{itemize}
	\item Wir kommentieren mit \#
	
	\begin{lstlisting}
	# Einfach so
	\end{lstlisting}

	\item Einzeiler
	\item Sinnvolle Kommentare
	\item Am Anfang jeder Python-Datei ein Kommentar, der den Inhalt beschreibt
\end{itemize}
\end{frame}

\begin{frame}[fragile]{Groß/Kleinschreibung und Einrückungen }

\begin{itemize}
	\item Fast alles wird klein geschreiben
	\item Es gibt Ebenen (durch Einrückungen = 4 Leerzeichen)
	\item Leerzeilen und Umbrüche sind nicht nötig, aber manchmal sinnvoll 
	
\end{itemize}
\end{frame}

\begin{frame}{Programm}
\begin{itemize}
	\item wird "von oben nach unten" ausgeführt
	\item kein automatisches springen (nach oben) oder neu starten
	\item ein komplettes Python Dokument mit allen Befehlen
\end{itemize}
\end{frame}

\begin{frame}{Funktion}
	\begin{itemize}
	\item wird ähnlich wie in der Mathematik verwendet (nur nicht mit Zahlen)
	\item eine Vielzahl von Befehlen (vorgefertigt oder selbstgeschrieben) zusammengefasst in einer bestimmten Schreibweise	
	\end{itemize}
\end{frame}

\defverbatim{\pythonenglish}{%
	\begin{lstlisting}[frame=none,basicstyle=\fontsize{9}{11}\ttfamily]	
	#Ein Beispielcode
	
	e2g_dict = {'a':'ein', 'is':'ist', 'test':'Test', 'this':'dies'}
	
	# Englisch nach Deutsch uebersetzen
	def translate(english):
		    return e2g_dict[english]
	
	esentence = 'this is a test'
	elist = esentence.split()
	glist = []
	
	for eword in elist:
		    glist = glist + [translate(eword)]	
		
	gsentence = " ".join(glist)
	print gsentence
	\end{lstlisting}
}
\begin{frame}[standout]
\pythonenglish
\end{frame}

