\begin{frame}[fragile]{Mit Dateien arbeiten}
	\begin{itemize}
		\item Dateien einlesen
		\item Inhalte bearbeiten
		\item Umgang mit verschiedenen Dateitypen
	\end{itemize}
\end{frame}

\begin{frame}[fragile]{Dateien lesen und schreiben}
Um Daten, die unsere Programme ausgeben bzw. benötigen, brauchen wir eine 
Möglichkeit, diese zu speichern.
\pause{}
\begin{lstlisting}
daten = open("daten.txt", "r")
for line in daten:
    print(line.rstrip())

daten.close()
\end{lstlisting}

\pause{}

\begin{lstlisting}
zahlen = [1, 2, 3]
daten = open("daten.txt", "w")
for zahl in zahlen:
    daten.write(str(zahl))

daten.close()
\end{lstlisting}
\end{frame}

\begin{frame}[fragile]{Dateien lesen und schreiben}
\begin{block}{Aufgabe}
Baue das Notenprogramm so um, dass die Noten beim Start des Programms aus einer Datei gelesen 
werden und nach Abschluss der Eingabe wieder dort hinein geschrieben werden.
\end{block}
\end{frame}


