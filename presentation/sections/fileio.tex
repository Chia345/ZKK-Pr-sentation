\begin{frame}[fragile]{Mit Dateien arbeiten}
	\begin{itemize}
		\item Dateien einlesen
		\item Inhalte bearbeiten
		\item Umgang mit txt
	\end{itemize}
\end{frame}

\begin{frame}[fragile]{Dateien lesen und schreiben}
Wir können Daten aus einer .txt-Datei einlesen, Inhalte verarbeiten und in einer gleichen oder anderen .txt-Datei wieder ausgeben.

Wichtiges:\\
 Liest Daten ein.
\begin{lstlisting}
daten = open("name.txt", "r")
\end{lstlisting}
 Öffnet eine Datei um Daten in sie schreiben zu können
\begin{lstlisting}
daten = open("name.txt", "w")
\end{lstlisting}
Schließt die Datei 
\begin{lstlisting}
daten.close()
\end{lstlisting}

\end{frame}


\begin{frame}[fragile]{Dateien lesen und schreiben}
 
Liest ein und entfernt unnötige Leerzeichen am Ende jeder Datenzeile
\begin{lstlisting}
zeilederdatei.rstrip()
\end{lstlisting}

 Schreibt in eine Datei
\begin{lstlisting}
datei.write(Element)
\end{lstlisting}
\end{frame}

\begin{frame}[fragile]{Dateien lesen und schreiben}
\begin{lstlisting}
# öffnet und speichert die Daten.
daten = open("daten.txt", "r")
for line in daten:
    print(line.rstrip())

daten.close()
\end{lstlisting}

\begin{lstlisting}
zahlen = [1, 2, 3]
daten = open("daten.txt", "w")
for zahl in zahlen:
    daten.write(str(zahl))

daten.close()
\end{lstlisting}
\end{frame}



\begin{frame}[fragile]{Dateien lesen und schreiben}
Schreiben geht nahezu analog. Zum Öffnen benutzt man ''w'' (für write) statt ''r''. Daten schreibt man in eine Datei mit der Methode 'write' des Dateiobjektes.

Beispiel:
\begin{lstlisting}
zahlen = [1, 2, 3]
daten = open("daten2.txt", "w")
for zahl in zahlen:
	daten.write(str(zahl))
daten.close()
\end{lstlisting}
\end{frame}


\begin{frame}[fragile]{Dateien lesen und schreiben}
Das Beispielprogramm liest aus der Datei daten3ein.txt ein, und schreibt in die Daten daten3aus.txt den doppelten Wert jeder einzelnen Zahl aus daten3ein.txt.


\begin{lstlisting}
daten_ein = open("daten3ein.txt", "r")
daten_aus = open("daten3aus.txt", "w")

for line in daten_ein:
	for number in line.strip():
		daten_aus.write(str(int(number) * 2) +"\n")

datenein.close()
datenaus.close()

Der String ''\n'' ist ein Zeilenumbruch.
\end{lstlisting}
\end{frame}