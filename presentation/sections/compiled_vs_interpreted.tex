\begin{frame}{Programmiersprachen}
    \begin{itemize}
        \item Unterscheidungsmerkmale
            \begin{itemize}
                \item Programmierparadigma: imperativ, funktional oder objektorientiert
                \item Typsicherheit
                \item kompiliert vs. interpretiert
                \item allgemein vs. domänenspezifisch
                \item hardwarenah vs. höhere Programmiersprachen
            \end{itemize}
    \end{itemize}
\end{frame}

\begin{frame}[fragile]{Programmiersprachen}
    \begin{itemize}
        \item Imperative Programmiersprachen: C/C++, C\#, Java \dots
        \item Funktionale Programmiersprachen: SQL, Haskell, Erlang, (Scala) \dots
        \item Objektorientierte Programmiersprachen: C++, C\#, Java, Javascript, PHP, Python \dots
    \end{itemize}
\end{frame}

\begin{frame}[fragile]{Imperative Sprachen (C/C++, C\#, Python, Java, \dots)}
    \begin{itemize}
        \item ältestes Programmierparadigma
        \item große Verbreitung in der Industrie
        \item besteht aus Befehlen (lat. imperare = befehlen)
        \item Abarbeiten der Befehle `Schritt für Schritt'
        \item sagt einem Computer, `wie' er etwas tun soll
    \end{itemize}
    \begin{lstlisting}
print("Hey, whats' up?")
sleep(3)
print("Learning Python right now")
sleep(2)
    \end{lstlisting}
    \begin{itemize}
        \item Verwendung
            \begin{itemize}
                \item `Standard-Software', hardwarenahe Entwicklung
            \end{itemize}
    \end{itemize}
\end{frame}

\begin{frame}[fragile]{Funktionale Sprachen (Haskell, Erlang, SQL, Lisp, \dots)}
    \begin{itemize}
       \item vergleichsweise modern
       \item sagt einem Computer, `was' das Ergebnis sein soll
       \item SELECT name FROM students WHERE major=‘law’ AND semester=‘1’;
       \item Verwendung
        \begin{itemize}
            \item akademische Zwecke
            \item sicherheitskritische und \dots
            \item hoch performante Anwendungen
        \end{itemize}
    \end{itemize}
    \begin{lstlisting}
square :: [Int] -> [Int]
square a = [2*x | x <- a]
    \end{lstlisting}
\end{frame}

\defverbatim{\impvsfunc}{%
\begin{lstlisting}[frame=none,numbers=none,basicstyle=\fontsize{15}{18}\ttfamily]
            x = x + 1
\end{lstlisting}
}
\begin{frame}[standout]
    \impvsfunc
\end{frame}

\begin{frame}[fragile]{Objektorientierte Sprachen (Java, Python, C++, C\#, \dots)}
    \begin{itemize}
       \item starke Verbreitung
       \item Abbilden der realen Welt der Dinge auf Objekte
       \item Klasse: Bauplan eines Objekts bestehend aus Eigenschaften (Attributen) und Methoden
       \item Vererbung möglich
       \item Verwendung
        \begin{itemize}
            \item Standard-Software
            \item Modellierung realer Projekte(Unternehmen, Mitarbeiter, Kunden, Waren, \dots)
            \item große Projekte ($\rightarrow$ Planung durch Klassendiagramme)
        \end{itemize}
    \end{itemize}
\end{frame}

\begin{frame}[fragile]{Objektorientierung: Beispiel}
    \begin{lstlisting}
class Konto:
    def __init__(self, name, nr):
        self.inhaber = name
        self.kontonummer = nr
        self.kontostand = 0
    def einzahlen(self, betrag):
        self.kontostand = kontostand + betrag
    def auszahlen(self, betrag):
        self.kontostand = kontostand - betrag
    def ueberweisen(self, ziel, betrag): 
        ziel.einzahlen(self.betrag)
        self.auszahlen(betrag)
    def kontostand(self):
        return self.kontostand

class Unternehmenskonto(Konto):
    def erhalteBonus(self, bonus):
        self.kontostand = kontostand + bonus
    \end{lstlisting}
\end{frame}

\begin{frame}[fragile]{Kompilierte und Interpretierte Sprachen}
    \begin{itemize}
        \item kompilierte Sprachen (Java, C/C++, C\#, \dots):
        \begin{itemize}
            \item Übersetzung des (kompletten) Programmcodes in Maschienencode
            \item dann Ausführung des Maschinencodes
        \end{itemize}
        \item interpretierte Sprachen (Python, Lisp, PHP, JavaScript, \dots):
        \begin{itemize}
            \item Übersetzung einer einzelnen Programmanweisung
            \item Ausführung dieser Anweisung
            \item Übersetzung der nächsten Anweisung
        \end{itemize}
    \end{itemize}
\end{frame}

\begin{frame}[fragile]{Hardwarenahe und höhere Sprachen}
\begin{itemize}
    \item hardwarenah: abhängig von der Bauweise des Prozessors
    \item höhere Sprachen: von der Bauweise abstrahiert (\texttt{print(), sleep()})
\end{itemize}
\begin{columns}
    \column{0.3\textwidth}
    \begin{lstlisting}
.START ST
  ST: MOV R1,#2
      MOV R2,#1
  M1: CMP R2,#20
      BGT M2
      MUL R1,R2
      INI R2
      JMP M1
  M2: JSR PRINT
      .END
    \end{lstlisting}
    \column{0.3\textwidth}
    \begin{lstlisting}

a = 2;
i = 1;
# compare i == 20
# if True, jump to M2
a = a*i;
i++;
# jump to M1
print(a)
    \end{lstlisting}
    \column{0.3\textwidth}
    \begin{lstlisting}
a = 2;
for i in range(1, 20) {


    a = a*i;
}
 
 
print(a);

    \end{lstlisting}
\end{columns}
\end{frame}

\begin{frame}{Populäre Programmiersprachen}
    \begin{itemize}
        \item C++
            \begin{itemize}
                \item imperativ, objektorientiert, typsicher, kompiliert, allgemein,
                höhere Sprache (dennoch hardwarenah)
                \item große Verwendung in hocheffizienten Systemen 
                (Betriebssysteme, Grafikberechnungen, Computerspiele, \dots)
                \item Erweiterung von C mit Objektorientierung
            \end{itemize}
        \item Java
            \begin{itemize}
                \item imperativ, objektorientiert, typsicher, kompiliert, allgemein
                \item im bayrischen Lehrplan und an vielen Universitäten 
                `erste' Sprache
                \item ebenfalls große Verbreitung
            \end{itemize}
        \item Python
            \begin{itemize}
                \item (imperativ), (funktional), objektorientiert, 
                dynamisch getypt, interpretiert, allgemein
                \item große Verbreitung auch gerade im akademischen Umfeld, 
                Web, Machine Learning und Data Science
            \end{itemize}
    \end{itemize}
\end{frame}

