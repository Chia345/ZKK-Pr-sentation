\section{Listen}

\begin{frame}[fragile]{Listen}
\begin{itemize}
	\item Einfache Listen
	\item Wichtige Listenfunktionen
	\item Mehrdimensionale Listen
\end{itemize}
\end{frame}

\begin{frame}[fragile]{Einfache Listen}
	\begin{itemize}
		\item Datenstruktur
		\item Speichert eine beliebige Anzahl an Elementen
		\item wir mit eckigen Klammern $[ ]$ dargestellt
		\item Trennung der Elemente durch Kommas
		\item Das erste Element ist der 0. Eintrag 
		
	\end{itemize}
\begin{lstlisting}
zahlen = [1, 2, 3, 4, 5, 6]
texte = ["Hallo", "Welt", ".", "Grundkurs", "Programmieren"]
gemischt =[True, "String", 3]
\end{lstlisting}
\end{frame}

\begin{frame}[fragile]{Einfacher Zugriff auf Elemente}
\begin{itemize}
	\item Stelle(index) wird benötigt	
\end{itemize}
\begin{lstlisting}
zahlen = [1, 2, 3, 4, 5, 6]

#Speichert den Inhalt der 2.Position in pos2
pos2 = zahlen[1]

#Speichert in die Liste an Position 2 die Zahl 99
zahlen[1] = 99
#aktualisierte Liste: [1, 99, 3, 4, 5, 6]
\end{lstlisting}
\end{frame}

\begin{frame}[fragile]{Mehrdimensionale Listen/ Listen in Listen}
\begin{itemize}
	\item Listen innerhalb listen	
\end{itemize}
\begin{lstlisting}
matrix = [[1, 2], [3, 4]]
\end{lstlisting}
Speichert den Inhalt der Position 1 der inneren Liste, die selbst auf Position 0 der {\"a}u{\ss}eren Liste ist in pos2. pos2 == 2
\begin{lstlisting}
pos2 = zahlen[0][1]
\end{lstlisting}
Speichert 99 an die Stelle [1][0]
\begin{lstlisting}
zahlen[1][0] = 99
#aktualisierte Liste: [[1, 2],[99, 3]]
\end{lstlisting}
\end{frame}



\begin{frame}[fragile]{Wichtige Funktionen}
Die Liste bietet eine große Anzahl an  Funktionen, die auf ihnen
ausgeführt werden können.

\begin{lstlisting}
liste = ["Grundkurs", "Programmieren", 42, "Pie", 3.14]

liste[2] = 99
len(liste)
liste.append("Passau")
liste.extend([4, 5, 3.14])
liste.insert(2, "Falke")
liste.count(3.14)
liste.index(3.14)
liste.remove(3.14)
liste.pop()
liste.reverse()
sum([1,3,5])
max([1,3,5])
\end{lstlisting}
\end{frame}


\begin{frame}[fragile]{Datenstrukturen: Listen}
\begin{block}{Aufgabe}
Versuche zu erraten, was die Ausgabe dieses Programms ist.
\end{block}

\begin{lstlisting}
liste_a = ["Hallo", "schoenes", "Wetter"]
liste_b = liste_a

liste_b[1] = "schlechtes"

print(liste_a[0], liste_a[1], liste_a[2])  
\end{lstlisting}
\pause{}
\begin{exampleblock}{Lösung}
    \texttt{Hallo schlechtes Wetter}
\end{exampleblock}
\end{frame}

\begin{frame}[fragile]{Datenstrukturen: Listen}

Schreibe ein Programm, dass drei Prüfungs-Noten einliest, in einer
Liste speichert und dir nach jeder Eingabe den Durchschnitt errechnet. 
\pause{}
\begin{exampleblock}{Lösung}
siehe Beamer
\end{exampleblock}
\end{frame}

