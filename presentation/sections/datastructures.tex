\section{Listen}

\begin{frame}[fragile]{Listen}
\begin{itemize}
	\item Einfache Listen
	\item Wichtige Listenfunktionen
	\item Mehrdimensionale Listen
\end{itemize}
\end{frame}

\begin{frame}[fragile]{Einfache Listen}
	\begin{itemize}
		\item Datenstruktur
		\item Speichert eine beliebige Anzahl an Elementen
		\item wir mit eckigen Klammern $[$ $]$ dargestellt
		\item Trennung der Elemente durch Kommas
		\item Das erste Element ist der 0. Eintrag 
		
	\end{itemize}
\begin{lstlisting}
zahlen = [1, 2, 3, 4, 5, 6]
texte = ["Hallo", "Welt", ".", "Grundkurs", "Programmieren"]
gemischt =[True, "String", 3]
\end{lstlisting}
\end{frame}

\begin{frame}[fragile]{Einfacher Zugriff auf Elemente}
\begin{lstlisting}
zahlen = [1, 2, 3, 4, 5, 6]
\end{lstlisting}
Ein Element aus der Liste lesen
\begin{lstlisting}
zahlen[Index]
zb. zahlen[0]
\end{lstlisting}

Ein Element aus der Liste lesen und in einer Variable speichern
\begin{lstlisting}
zahl = zahlen[Index]
zahl = zahlen[0]
zahl == 1
\end{lstlisting}

Speichert den Wert an der entsprechenden Position
\begin{lstlisting}
zahlen[Index] = wert
zahlen[0] = 99
zahlen == [99, 2, 3, 4, 5, 6]
\end{lstlisting}

\end{frame}

\begin{frame}[fragile]{Listenfunktionen}
Seien s = [1, 2] und t = [3, 4] zwei Listen
\begin{itemize}
\item s + t 
\begin{lstlisting}
s + t == [1, 2, 3, 4]
\end{lstlisting}
\item len(s) 
\begin{lstlisting}
len(s) == 2
\end{lstlisting}
\item min(s)
\begin{lstlisting}
min(s) == 1
\end{lstlisting}
\item max(s) 
\begin{lstlisting}
max(s) == 2
\end{lstlisting}
\item sum(s) 
\begin{lstlisting}
sum(s) == 3
\end{lstlisting}
\end{itemize}
\end{frame}



\begin{frame}[fragile]{Listen}
zahlen = [5, 9, 2]\\
zahlen2 = [4, 1, 0]
\begin{itemize}
\item Aufgabe 1\\
''Addiere'' die beiden Listen und speichere das Ergebnis wieder in `zahlen`

\item Aufgabe \\
Lasse die Länge der Liste `zahlen` ausgeben. 

\item Aufgabe \\
Lasse das Minimum der Liste `zahlen` ausgeben

\item Aufgabe \\
Lasse das Maximum der Liste `zahlen` ausgeben

\item Aufgabe \\
Lasse die Summe der Werte in der Liste `zahlen` ausgeben
\end{itemize}
\end{frame}

\begin{frame}[fragile]{Listenfunktionen}
Sei s = [1, 2, 3, 4, 5]
\begin{itemize}
	\item s.append(x)
	\begin{lstlisting}
s.append(6) => s == [1, 2, 3, 4, 5, 6]
	\end{lstlisting}
	\item s.remove(x)
	\begin{lstlisting}
s.remove(3) => s == [1, 4, 5, 6]
	\end{lstlisting}
	\item s.insert(position, object)
	\begin{lstlisting}
s.insert(2, 8) => s == [1, 4, 8, 5, 6]
	\end{lstlisting}
\end{itemize}
\end{frame}

\begin{frame}[fragile]{Listen}
liste = [1, ''Passau'', 4, 6, ''Berlin``]
\begin{itemize}
\item Aufgabe \\
Füge ''Hamburg'' am Ende der liste1 an und gebe die Liste aus

\item Aufgabe \\
Entferne ''Berlin'' aus der Liste und gebe die Liste neu aus. 

\item Aufgabe 3\\
Füge 42 am Index 0 ein. Gebe die Liste erneut aus.
\end{itemize}
\end{frame}

\begin{frame}[fragile]{Schleifen über Listen}
For-Schleifen für Listen

Bisher:
\begin{lstlisting}
for i in range(anfang, ende):
	....
\end{lstlisting}

$range(anfang, ende)$ kann ersetzt werden durch eine Liste.
Dh. sobald wir eine Liste haben, können wir mit der for-Schleife einmal die Liste durchgehen

Bsp. 
\begin{lstlisting}
zahlen = [1,4,6]
for zahl in zahlen:
print(zahl)
\end{lstlisting}
\end{frame}

\begin{frame}[fragile]{Listen - Übung}
Schreibe ein Programm, das drei Prüfungs-Noten einliest, in einer
Liste speichert und dir nach jeder Eingabe den Durchschnitt errechnet. 
\pause{}
\begin{exampleblock}{Lösung}
siehe Beamer
\end{exampleblock}
\end{frame}

