\section{Funktionen}
\begin{frame}[fragile]{Funktionen}
\begin{itemize}
	\item Funktionen ohne Parameter
	\item Funktionen mit Parameter
	\item Rückgabewert 
\end{itemize}
\end{frame}

\begin{frame}[fragile]{Funktionen}
	\begin{itemize}
		\item Fasst mehrere sinnvolle Befehle zusammen
		\item Kann später im Code aufgerufen werden
		\item ''Auslagerung von Code'' (Beispielsweise, weil er häufiger verwendet werden soll)
	\end{itemize}
\end{frame}

\begin{frame}[fragile]{Funktionen ohne Parameter}
Jedes mal, wenn die Funktion name() im späteren Verlauf aufgerufen wird, wird der Programmcode ausgeführt. \\
Wichtig: Die Definition der Funktion ist keine Ausführung der Funktion.
\begin{lstlisting}
def name():
	#Programmcode
\end{lstlisting}

Beispiel
\begin{lstlisting}
def greet():
    print("Hey!")
    print("How are you?")
\end{lstlisting}
\end{frame}


\begin{frame}[fragile]{Funktionen mit Parameter}
Es werden ein oder mehrere Parameter übergeben. \\
Bei jedem Aufruf muss die Anzahl an Parametern übergeben werden.
\begin{lstlisting}
def name(a, b, c, ....):
	#Programmcode
\end{lstlisting}

\begin{lstlisting}
def greet(name):
	print("Hey!" + name)
	print("How are you?")
\end{lstlisting}
\end{frame}


\begin{frame}[fragile]{Funktionen mit Parameter und Rückgabewert}
Die Funktion bekommt mehrere Parameter übergeben a,b,c,... 
Führt den in der Funktion enthaltenen Code aus und gibt einen Wert zurück.
\begin{lstlisting}
def name(a, b, c, ....):
	#Programmcode
	return wert
\end{lstlisting}

\begin{lstlisting}
#gibt die Summe zweier Werte zurück
def sum(wert1, wert2):
	return wert1 + wert2	
\end{lstlisting}
\end{frame}

\begin{frame}[fragile]{Übungsaufgabe}
Schreibe eine Funktion, die die Summe aus 3 Zahlen bildet.\\
Lese 3 Zahlen jeweils ein und speichere sie ab.\\
Rufe die Funktion mit den 3 Zahlen als Parameter auf, speichere das Ergebnis und gebe es aus.

\end{frame}


\begin{frame}[fragile]{Übungsaufgabe}
Schreibe eine Funktion die 2 Parameter erhält und das Produkt ausrechnet und zurückgibt.

Lese 2 Zahlen ein mit input(''Zahl: '') und speichere sie jeweils ab. Rufe die Funktion auf und übergebe die beiden Zahlen.

Gebe das Ergebnis mit print(produkt) aus


\end{frame}







