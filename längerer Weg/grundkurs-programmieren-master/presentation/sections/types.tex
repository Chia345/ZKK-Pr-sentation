\section{Datentypen}

\begin{frame}[fragile]{Datentypen}
Lernziele
\begin{itemize}
	\item Die wichtigsten Datentypen kennenlernen
	\item Diese ausgeben können
	\item Datentypen in andere Datentypen umwandeln
\end{itemize}
\end{frame}

\begin{frame}[fragile]{Erste wichtige Funktion: print()}

	print():\\
	Gibt alles innerhalb der Klammern auf die aus.

\begin{lstlisting}
print("Hallo")
print(1)
print(1+2)
\end{lstlisting}
\end{frame}



%Einführung String
\begin{frame}[fragile]{String}
\begin{itemize}
    \item \texttt{String, str}: 
     	\begin{itemize}
     		\item ist eine Zeichenkette
     		\item wird in '' '' geschrieben
     	\end{itemize}
     \end{itemize}
    \begin{lstlisting}
"Ich bin vom Typ String, eine Reihe von Zeichen"
"1"
" "
    \end{lstlisting}
\end{frame}

%Beispielaufgabe für String
\begin{frame}[fragile]{Hello World}
\begin{lstlisting}
print("Hello World!")
\end{lstlisting}
\begin{itemize}
	\item gibt den Text (String) ''Hello World!'' aus
\end{itemize}
\begin{exampleblock}{Glückwunsch}
	Ihr habt gerade euer erstes Codeprogramm geschrieben!
\end{exampleblock}
\end{frame}


%Einfürhung Integer udn Float
\begin{frame}[fragile]{Zahlen}

	 \texttt{Integer, int}: 
	\begin{itemize}
		\item ist eine ganze Zahl
	\end{itemize}
	\begin{lstlisting}
-1
2
3
...
	\end{lstlisting}
	 \texttt{Float, float}:
    	\begin{itemize}
    	\item ist eine Gleitkommazahl
    \end{itemize}
    \begin{lstlisting}
3.1415
3.0
-2.3
    \end{lstlisting}

\end{frame}


%Einführung Boolean
\begin{frame}[fragile]{Boolean}
\texttt{Boolean, bool}:
\begin{itemize}
	\item Wahrheitswert
\end{itemize}

\begin{lstlisting}
True
False
\end{lstlisting}

\end{frame}

\begin{frame}[fragile]{Typumwandlung}
\begin{itemize}
	\item \lstinline{int(...)}: Castet zu int.
	\item \lstinline{float(...)}: Castet zu int.
	\item \lstinline{str(...)}: Castet zu String.
	
\end{itemize}
Wandle um und gebe mit print() aus	
\begin{itemize}
	\item 5 zu ''5''
	\item ''5.0'' zu 5.0
	\item ''Hallo'' + 5 zu ''Hallo 5'' 
\end{itemize}
\pause{}
\begin{lstlisting}
print(str(5))
print(float("5.0"))
print("Hallo" + " " + str(5))
\end{lstlisting}
\end{frame}





